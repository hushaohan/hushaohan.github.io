\section{\sc Research} {\bf University of Illinois at Urbana-Champaign} \hfill August 2011--Present\\
Advisor: Tarek F. Abdelzaher
\begin{itemize}
\item \textbf{Semantic Quality-Aware Information Delivery under Resource Constraints}\\ Funded by U.S. Army Research Laboratory as part of the Network Science Collaborative Technology Alliance, in collaboration with University of Southern California, the City University of New York, Raytheon BBN Technologies, and U.S. Army Research Laboratory, this project aims at building algorithms and systems that are geared towards crowd-sensing under resource-limited post-disaster scenarios, and optimize network resource consumptions while serving multiple concurrent queries with timely requirements (resulting papers published in \hyperlink{hu2015dcoss}{\textsc{DCOSS 2015}} and \hyperlink{hu2015rtss}{\textsc{RTSS 2015}}).

\item \textbf{eNav: Smartphone-based Energy-Efficient Vehicular Navigation}\\ In collaboration with Microsoft Research, this project aims at providing low-power reliable real-time road navigation services by taking advantage of the phone's on-board MEMS sensors (accelerometer, gyroscope, etc) to minimize the need for GPS localization, in order to best preserve phone battery to ultimately eliminate the need for carrying extra hardware (car charger, cable, car mount) or dedicated GPS units for navigation (resulting poster received the \hyperlink{hu2014ipsn}{IPSN 2014 best poster award}, full paper published in \hyperlink{hu2015ubicomp}{\textsc{UbiComp 2015}}).

\item \textbf{SmartRoad: Participatory Road Sensing} \\
Funded by NSF, Siebel Foundation, this project aims at providing the next generation road services by taking advantage of sensor-packed mobile phones in a vehicular environment. Various applications/services include:
\begin{itemize}
\item Vehicular mobile sensing and communication system, for which we implement and deploy a participatory vehicle mobile sensing system, and study and analyze how to most efficiently carry out car-to-car data sharing when two cars meet, and how to balance among available transmission channels (WiFi, cellular, etc) for server offloading (resulting papers published in \hyperlink{liu2013infocom}{\textsc{Infocom 2013}}, \hyperlink{hu2014infocom}{\textsc{2014}}, and \hyperlink{hu2015tmc}{\textsc{TMC 2015}}).

\item Traffic regulator detection and identification, for which we design, build, and deploy a participatory sensing system that collects and intelligently handles noisy crowd-sourced data, and robustly, efficiently and effectively detects and identifies traffic regulators, such as stop signs and traffic lights, under various energy/bandwidth conditions (resulting work published in \hyperlink{hu2013ipsn}{\textsc{IPSN 2013}}, \hyperlink{wang2013rtss}{\textsc{RTSS 2013}}, \hyperlink{wang2015jrts}{\textsc{Journal of Real-Time Systems}}, and \hyperlink{hu2015tosn}{\textsc{ACM TOSN}}).

\item GreenGPS: Fuel-efficient mapping and routing, which aims at providing, in addition to the shortest and fastest, the most fuel-efficient route between any pair of source and destination points on the map. Currently we are in the process of progressively deploying our system to the UIUC Facility \& Service department (resulting paper published in \hyperlink{saremi2015tmc}{\textsc{IEEE TMC}}).
\end{itemize}

\item \textbf{Resource-Efficient Data Classification in Distributed Sensing Systems} \\
This project aims at building a resource-efficient system to classify sensory data distributed over a large number of sensor nodes under stringent network resources. We focus on two aspects of the sensor network nodes: data reliability and data redundancy, where reliability implies the degree to which a sensor node contributes to the classification mission, and redundancy represents the information overlap among different sensor nodes. We formulate and solve an optimization problem that maximizes the reliability of sensory data while eliminating their redundancies under the constraint of network resources (resulting papers published in \hyperlink{su2012rtss}{\textsc{RTSS 2012}} and \hyperlink{su2014rtss}{\textsc{2014}}).

\item \textbf{NDN: Named Data Networking} \\
Funded by NSF, in collaboration with 10 other Universities and PARC. NDN is a recently proposed next-generation networking framework design, which aims at basing addresses on data names instead of machine locations. We aim at taking advantage of the ``named data'' nature of the NDN network and putting focus on maximizing information coverage rather than just data throughput when designing communication mechanisms under resource constraints (resulting paper published in \hyperlink{wang2013icccn}{\textsc{ICCCN 2013}}).

\end{itemize}

{\bf Dartmouth College} \hfill September 2009--June 2011\\
Advisor: Andrew T. Campbell
\begin{itemize}
\item \textbf{Large Scale Activity Recognition using Community Similarity Networks} \\
This work targets on building a mobile phone human activity recognition system that can accurately carry out the recognition tasks on diverse populations by exploiting crowd-sourced sensor data, incorporating inter-person similarity measurements, and automatically personalizing classifiers with data contributed from other similar users (resulting paper published in \hyperlink{lane2011ubicomp}{\textsc{UbiComp 2011}}).

\item \textbf{Bridging Consumer Neural-Headset with Mobile Platforms} \\
In this work, we build NeuroPhone, a mobile system that is driven by neural signals, using a newly available off-the-shelf neural headset, for hands-free, silent and effortless human-mobile interaction. We build an address book dialing app on iPhone, which natively runs a lightweight classifier that extracts and detects the P300 brain potential from the EEG signal wirelessly transmitted by the headset (resulting paper published in \hyperlink{campbell2010mobiheld}{\textsc{MobiHeld 2010}}).
\end{itemize}

{\bf University of Massachusetts at Amherst} \hfill January 2008--May 2008 \\
Advisor: Andrew McCallum
\begin{itemize}
\item \textbf{Resource-bounded Information Extraction} \\
This work aims to design a general framework for the task of extracting specific information ``on demand'' from a large corpus such as the Web under resource constraints. Given a database with missing or uncertain information, our system automatically formulates queries, issues them to a search interface, selects a subset of the documents, extracts the required information from them, and fills the missing values in the original database (resulting paper published in \hyperlink{kanani2010pakdd}{\textsc{PAKDD 2010}}).
\end{itemize}
